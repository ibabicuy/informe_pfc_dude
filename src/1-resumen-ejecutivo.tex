\chapter{Resumen Ejecutivo}

\section{Antecedentes}

Este proyecto nace por el interés en la automatización de los objetos cotidianos. En la actualidad tanto los componentes electrónicos como las plataformas y protocolos de IoT han avanzado a tal punto que es posible su implementación a un costo accesible para las masas. Se desea que este proyecto contribuya en esta proximidad creciente y hacer de las casas inteligentes un sueño factible para cualquier interesado.

\section{Objetivos}

Se creó un sistema de domótica altamente adaptable, de fácil instalación y uso. El mismo provee una base de de hardware y software capaces de configurarse a través de la aplicación, y permitir así que nuevos desarrolladores se enfoquen en la construcción de interfaces que consuman sus servicios. Nuestro objetivo final fue lograr un producto con un costo por el cual sea posible automatizar la totalidad de los elementos electrónicos de un hogar, oficina o empresa, a un precio varias veces menor que la competencia y brindando un servicio que compita con las soluciones actuales, dando la opción de no utilizar aplicaciones o asistentes virtuales de empresas que registren los comportamientos de los usuarios.

\section{Metodología}

Se programó los dispositivos Sonoff, y se utilizó un Raspberry Pi como plataforma central para contener los servicios de coordinación y configuración, permitiendo que el proceso de instalación y uso sea lo más simple e intuitivo posible. Se proveen dos métodos de interacción con el sistema, siendo estos reemplazables en un futuro con soluciones de terceros como Google Home o Alexa. 

\section{Objetivos cumplidos}

Se logró el cometido de crear un sistema cuyas partes son configurables con una mínima intervención.