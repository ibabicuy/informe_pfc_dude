\chapter{Conclusiones}

Se cumplió con todos los objetivos previamente establecidos. El sistema cuenta con una API que puede ser llamada por distintas interfaces de forma transparente para el usuario. Esto puede verse mediante la utilización del asistente por voz Mycroft, cuya función es interpretar comandos de voz para llamar a la API del sistema. Es un sistema muy flexible, el  lugar de Mycroft podría estar ocupado por sensores de movimiento, sensores de calor, etc. Esta flexibilidad hace al sistema atractivo para empresas que necesitan una solución a medida.

A su vez, el respeto por la privacidad del usuario es clave en el sistema. Muchas soluciones de domótica hacen pasar sus request por servidores que registran las acciones del usuario. De esta forma pueden obtener datos como el horario en que el usuario está en su casa, cuando duerme. También se debe destacar que la seguridad es un punto fuerte en este sistema. El poco poder de procesamiento en los dispositivos de IoT suele ser un problema para la seguridad de los mismos, sin embargo, se eligió una arquitectura que aunque hizo el desarrollo más complejo, esto se tradujo a seguridad.

Respecto a la competitividad del producto, vale la pena mencionar que una de las motivaciones para la elección de este tema fué el interés por la domótica y las ansias de poseer este tipo de tecnología. El problema con las opciones a disposición son varios, aunque la mayor barrera resultaba ser los elevados precios de los productos. Por ejemplo los productos Hue de Phillips promocionados por el asistente virtual de Google, cuyos precios más modestos son de USD 50 por un pack de 4 luces blancas de intensidad fija.

Otro punto a tener en cuenta es que este método donde la bombilla contiene la lógica resulta en que un artefacto luminoso con varias bombillas aumente sustancialmente el gasto necesario para automatizar un hogar u oficina. Siendo también necesario el que el artefacto en cuestión sea compatible con la conexión de la bombilla, quitando libertad en la elección de las mismas y teniendo que reemplazar las previamente instaladas.

Nuestra solución aporta la filosofía de automatizar elementos previamente existentes y no se limita a artefactos luminosos, ya que es posible controlar termotanques u otros electrodomésticos como calderas eléctricas.

Sumado a esto se buscó que el sistema no dificulte el funcionamiento actual de las instalaciones, ya que debido a que el dispositivo Sonoff modificado es conectado a los interruptores eléctricos colocado en las paredes, el usuario puede controlar el dispositivo tanto analógicamente como con nuestro sistema sin problemas. En el caso presentado por la mayoría de la competencia, es imposible encender una luz cuya llave no brinde corriente a la misma, siendo que nuestro sistema siempre se encuentra encendido, sin importar el estado del relay que controla el estado de la corriente.

Asimismo, el uso de una librería basada en la arquitectura mesh hace posible que el sistema pueda usarse en lugares con alcance de la red WiFi limitada. Los estudiantes sienten que haber trabajado con este tipo de arquitecturas les sirvió mucho desde el punto de vista técnico, dado que este tipo de arquitectura es muy novedosa y trae muchos beneficios.

Este proyecto fue muy provechoso para los alumnos. Estos pudieron trabajar con tecnologías y lenguajes de programación con las cuales no habían tenido oportunidad durante la carrera. Esto representó un gran desafío, pero a la vez el resultado superó lo que ellos esperaban cuando empezaron. El tener que adaptar muchas de las librerías que se utilizaron para la necesidades del proyecto hizo que los estudiantes tuvieran que estar codo con los programadores de las mismas tanto buscando errores como codificando. Incluso se aportó soluciones a problemas y se aportaron ideas que quedan como referencia para las personas que elijan estas librerías. Esta experiencia profesional es invaluable a la hora de salir al mercado laboral.
